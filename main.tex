\documentclass[10pt,spanish,xcolor={svgnames}]{beamer}
%%documentclass[notes,spanish,10pt]{beamer}       % print frame + notes

% TILDES Y DEMÁS EN ESPAÑOL
\usepackage[spanish]{babel}
\usepackage{fontspec} 
\setmainfont{Fira Sans Ultralight} 
\setsansfont{Fira Sans Ultralight} 
\setmonofont{Fira Mono Regular} 
\usetheme[numbering=fraction, progressbar=head]{metropolis}
\usepackage{appendixnumberbeamer}
%%	NUMEROS COMO FRACCIONS 4/20 Y BARRA DE PROGRESO EN CADA DIAPOSITIVA (TOO MUCH?)
%\setsansfont[BoldFont={Fira Sans SemiBold}]{Fira Sans Book} %TEXTO QUE SE LEE MEJOR PARA UNA SALA GRANDE Y PROYECTOR CON POCA POTENCIA

\usepackage{booktabs}
\usepackage[scale=2]{ccicons}

\usepackage{pgfplots}
\usepgfplotslibrary{dateplot}
\usepackage{xspace}
\newcommand{\themename}{\textbf{\textsc{metropolis}}\xspace}
% \setbeamertemplate{footline}[frame number] % PARA PONER EL NUMERO DE PAGINA EN FRACCIÓN: 7/20

% Comando para poder añadir la fuente en mis figuras
\newcommand{\source}[1]{{\textbf{Fuente}: {#1}} }

\usepackage{hyperref} % LINKS


\usepackage{pgfpages} %PARA LAS NOTAS
\setbeameroption{show notes}                        
\setbeameroption{show notes on second screen=right} %% en terminal, $: pdfpc main.pdf --notes=right
%https://bugs.kde.org/show_bug.cgi?id=152585


\setbeamertemplate{note page}{\pagecolor{yellow!5}\insertnote}% MUY IMPORTANTE, ASÍ MUESTRA TAN SOLO UNA NOTA EN  BLANCO EN VEZ DE MOSTRAR UNA DIAPO DE NOTA ESTRUCTURADA (UNA PARTE ES LA NOTA EN SI, OTRA LA SIGUIENTE DIAPO Y OTRA EL ARBOL (SECCION->SUBSECCION->TITLE _RAME) Y NO QUEREMOS ESO.
%\usepackage{palatino}  % SI DESCOMENTO ESTO SE ME JOROBAN TODAS LAS FUENTES DE LAS DIAPOS. NO TOCAR LA FUENTE DE LAS NOTAS.



%para que funcionen bien las notas, pero mandarlo a tomar por saco, sin ello no se notaba diferencia
%\makeatletter\def\beamer@framenotesbegin{%atbeginningofslide
%\usebeamercolor[fg]{normal text}\gdef\beamer@noteitems{}%
%\gdef\beamer@notes{}%
%}\makeatother


\usepackage{dirtytalk} % PARA CITAR, CITAS, QUOTES.

%TODOs notas:
%\usepackage[colorinlistoftodos,prependcaption,textsize=small]{todonotes} %% PARA QUE NO APAREZCAN [disable]


%\usepackage{xcolor} % CAMBIAR TEXTO DE COLOR (PARA LAS NOTAS)
\usepackage{wrapfig} % FIGURAS  A UN LADO, CON TEXTO AL OTRO.





%%%%%%%%%%%%%%%%%%%%%%%%%%%%%%%%%%%%%%%%%%%%%%%%%%%%
%%%%%%%%%%%%%%%%%%%%%%%%%%%%%%%%%%%%%%%%%%%%%%%%%%%%
%%%%%%%%%%%%%%%%%%%%%%%%%%%%%%%%%%%%%%%%%%%%%%%%%%%%
%%%%%%%%%%%%%%%%%%%%%%%%%%%%%%%%%%%%%%%%%%%%%%%%%%%%
%%%%%%%%%%%%%%%%%%%%%%%%%%%%%%%%%%%%%%%%%%%%%%%%%%%%
%%%%%%%%%%%%%%%%%%%%%%%%%%%%%%%%%%%%%%%%%%%%%%%%%%%%
%%%%%%%%%%%%%%%%%%%%%%%%%%%%%%%%%%%%%%%%%%%%%%%%%%%%
%%%%%%%%%%%%%%%%%%%%%%%%%%%%%%%%%%%%%%%%%%%%%%%%%%%%
%%%%%%%%%%%%%%%%%%%%%%%%%%%%%%%%%%%%%%%%%%%%%%%%%%%%
%%%%%%%%%%%%%%%%%%%%%%%%%%%%%%%%%%%%%%%%%%%%%%%%%%%%
%%%%%%%%%%%%%%%%%%%%%%%%%%%%%%%%%%%%%%%%%%%%%%%%%%%%
%%%%%%%%%%%%%%%%%%%%%%%%%%%%%%%%%%%%%%%%%%%%%%%%
\usepackage{framed}
\usepackage{ifxetex,ifluatex}
\usepackage{etoolbox} 
% conditional for xetex or luatex
\newif\ifxetexorluatex
\ifxetex
  \xetexorluatextrue
\else
  \ifluatex
    \xetexorluatextrue
  \else
    \xetexorluatexfalse
  \fi
\fi
%
\ifxetexorluatex%
  \usepackage{fontspec}
  \setmainfont{Fira Sans Ultralight} % or use \setmainfont to choose any font on your system
  \newfontfamily\quotefont[Ligatures=TeX]{Fira Sans Ultralight} % selects Libertine as the quote font
\fi

\newcommand*\quotesize{60} % if quote size changes, need a way to make shifts relative
% Make commands for the quotes
\newcommand*{\openquote}
   {\tikz[remember picture,overlay,xshift=-4ex,yshift=-2.5ex]
   \node (OQ) {\quotefont\fontsize{\quotesize}{\quotesize}\selectfont``};\kern0pt}

\newcommand*{\closequote}[1]
  {\tikz[remember picture,overlay,xshift=4ex,yshift={#1}]
   \node (CQ) {\quotefont\fontsize{\quotesize}{\quotesize}\selectfont''};}

% select a colour for the shading
\colorlet{shadecolor}{Azure}

\newcommand*\shadedauthorformat{\emph} % define format for the author argument

% Now a command to allow left, right and centre alignment of the author
\newcommand*\authoralign[1]{%
  \if#1l
    \def\authorfill{}\def\quotefill{\hfill}
  \else
    \if#1r
      \def\authorfill{\hfill}\def\quotefill{}
    \else
      \if#1c
        \gdef\authorfill{\hfill}\def\quotefill{\hfill}
      \else\typeout{Invalid option}
      \fi
    \fi
  \fi}
% wrap everything in its own environment which takes one argument (author) and one optional argument
% specifying the alignment [l, r or c]
%
\newenvironment{shadequote}[2][l]%
{\authoralign{#1}
\ifblank{#2}
   {\def\shadequoteauthor{}\def\yshift{-2ex}\def\quotefill{\hfill}}
   {\def\shadequoteauthor{\par\authorfill\shadedauthorformat{#2}}\def\yshift{2ex}}
\begin{snugshade}\begin{quote}\openquote}
{\shadequoteauthor\quotefill\closequote{\yshift}\end{quote}\end{snugshade}}

%%%%%%%%%%%%%%%%%%%%%%%%%%%%%%%%%%%%%%%%%%%%%%%%%%%%
%%%%%%%%%%%%%%%%%%%%%%%%%%%%%%%%%%%%%%%%%%%%%%%%%%%%
%%%%%%%%%%%%%%%%%%%%%%%%%%%%%%%%%%%%%%%%%%%%%%%%%%%%
%%%%%%%%%%%%%%%%%%%%%%%%%%%%%%%%%%%%%%%%%%%%%%%%%%%%
%%%%%%%%%%%%%%%%%%%%%%%%%%%%%%%%%%%%%%%%%%%%%%%%%%%%
%%%%%%%%%%%%%%%%%%%%%%%%%%%%%%%%%%%%%%%%%%%%%%%%%%%%
%%%%%%%%%%%%%%%%%%%%%%%%%%%%%%%%%%%%%%%%%%%%%%%%%%%%
%%%%%%%%%%%%%%%%%%%%%%%%%%%%%%%%%%%%%%%%%%%%%%%%%%%%
%%%%%%%%%%%%%%%%%%%%%%%%%%%%%%%%%%%%%%%%%%%%%%%%%%%%
%%%%%%%%%%%%%%%%%%%%%%%%%%%%%%%%%%%%%%%%%%%%%%%%%%%%
%%%%%%%%%%%%%%%%%%%%%%%%%%%%%%%%%%%%%%%%%%%%%%%%%%%%
%%%%%%%%%%%%%%%%%%%%%%%%%%%%%%%%%%%%%%%%%%%%%%%%%%%%
%%%%%%%%%%%%%%%%%%%%%%%%%%%%%%%%%%%%%%%%%%%%%%%%%%%%


%%%%%%%%%%%%%%%%%%%%%%%%%%%%%%%%%%%%%%%%%%%%%%%%%%%%%%%%%%%%%%%%%%%%%%%%%%%%%%%%
%%%%%%%%%%%%%%%%%%%%%%%%%%%%%%%%%%%%%%%%%%%%%%%%%%%%%%%%%%%%%%%%%%%%%%%%%%%%%%%%
%%%%%%%%%%%%%%%%%%%%%%%%%%%%%%%%%%%%%%%%%%%%%%%%%%%%%%%%%%%%%%%%%%%%%%%%%%%%%%%%
% DATOS DE LA PORTADA Y DEMÁS
%%%%%%%%%%%%%%%%%%%%%%%%%%%%%%%%%%%%%%%%%%%%%%%%%%%%%%%%%%%%%%%%%%%%%%%%%%%%%%%%
%%%%%%%%%%%%%%%%%%%%%%%%%%%%%%%%%%%%%%%%%%%%%%%%%%%%%%%%%%%%%%%%%%%%%%%%%%%%%%%%
%%%%%%%%%%%%%%%%%%%%%%%%%%%%%%%%%%%%%%%%%%%%%%%%%%%%%%%%%%%%%%%%%%%%%%%%%%%%%%%%


\title{Estudio de técnicas de Ingeniería de Tráfico basadas en SDN}
\subtitle{Study of SDN Traffic Engineering techniques}

%\author{Betegon Garcia, Miguel}
%\institute{Center for modern beamer themes}
\author[M. Betegón]{ Betegón García, Miguel\inst{1} \\ {\ttfamily miguel.betegon@alumnos.unican.es}} 

\institute[UC] % (optional)
{
	\inst{1}%
	Grado en Ingeniería de Tecnologías de Telecomunicación
}
%\date{\today}
\date{Junio, 2018}
\logo{
	\includegraphics[width=1.1cm,keepaspectratio]{figuras/git.png}
}
\titlegraphic{
	
	\includegraphics[width=3.5em]{figuras/uc.png}\hfill
	\includegraphics[width=5em]{figuras/uccc_no_letras.png}\hfill
	\includegraphics[width=3.2em]{figuras/git.png}
}

\setbeamertemplate{title page}{
	\begin{minipage}[b][\paperheight]{\textwidth}
		\vfill%
		\ifx\inserttitle\@empty\else\usebeamertemplate*{title}\fi
		\ifx\insertsubtitle\@empty\else\usebeamertemplate*{subtitle}\fi
		\usebeamertemplate*{title separator}
		\ifx\beamer@shortauthor\@empty\else\usebeamertemplate*{author}\fi
		\ifx\insertdate\@empty\else\usebeamertemplate*{date}\fi
		\ifx\insertinstitute\@empty\else\usebeamertemplate*{institute}\fi
		\vfill
		\ifx\inserttitlegraphic\@empty\else\inserttitlegraphic\fi
		\vspace*{1cm}
	\end{minipage}
}

%	\titlegraphic{\vspace{-1.6em}\includegraphics[height=1.5cm]{logo.pdf}}
\pgfplotsset{width=10cm,compat=1.9}
\usepackage{pgfplotstable}


%% ---- COMIENZA EL DOCUMENTO ----
%%%%%%%%%%%%%%%%%%%%%%%%%%%%%%%%%%%%%%%%%%%%%%%%%%%%%%%%%%%%%%%%%%%%%%%%%%%%%%%%%%%%%%%%%%%%%%%%%%%%%%%%%%%%%%%%%%%%%%%%%%%%%%%%%%%%%%%%%%%%%%%%%%%%%%%%%%%%%%%%%%%%%%%%%%%%%%%%%%%%%%%%%%%%%%%%%%%%%%%%%%%%%%%%%%%%%%%%%%%%%%%%%%%%%%%%%%%%%%%%%%%%%%%%%%%%%%%%%%%%%%%%%%%%%%%%%%%%%%%%%%%%%%%%%%%%%%%%%%%%%%%%%%%%%%%%%%%%%%%%%%%%%%%%%%%%%%%%%%%%%%%%%%%%%%%%%%%%%%%%%%%%%%%%%%%%%%%%%%%%%%%%%%%%%%%%%%%%%%%%%%%%%%%%%%%%%%%%%%%%%%%%%%%%%%%%%%%%%%%%%%%%%%%%%%%%%%%%%%%%%%%%%%%%%%%%%%%%%%%%%%%%%%%%%%%%%%%%%%%%%%%%%%%%%%%%%%%%%%%%%%%%%%%%%%%%%%%%%%%%%%%%%%%%%%%%%%%%%%%%%%%%%%%%%%%%%%%%%%%%%%%%%%%%%%%%%%%%%%%%%%%%%%%%%%%%%%%%%%%%%%%%%%%%%%%%%%%%%%%%%%%%%%%%%%%%%%%%%%%%%%%%%%%%%%%%%%%%%%%%%%%%%%%%%%%%%%%%%%%%%%%%%%%%%%%%%%%%%%%%%%%%%%%%%%%%%%%%%%%%%%%%%%%%%%%%%%%%%%%%%%%%%%%%%%%%%%%%%%%%%%%%%%%%%%%%%%%%%%%%%%%%%%
\begin{document}
\begin{frame}
  \titlepage
\note[item]{\textcolor{red}{COMPROBAR QUE FUNCIONA BIEN CON UNA SEGUNDA PANTALLA, QUE SE MUESTRAN LAS NOTAS EN UNA Y LA PRESENTACIÓN EN OTRA.}}
		\note[item]{\textcolor{red}{COMPROBARLO CON LAS PANTALLAS DE LA PRESENTACIÓN (MI ORDE NO TIENE VGA, PREGUNTAR SI HAY HDMI Y LLEVAR EL MIO POR SI ACASO)}}
		\note[item]{\textcolor{red}{EN LA TERMINAL, \$: pdfpc main.pdf - -notes=right}}
		\note[item]{\textcolor{red}{https://bugs.kde.org/show\_bug.cgi?id=152585}}
		\note[item]{\textcolor{red}{ESTOY COMPILANDO DESDE OVERLEAF CON XUALATEX. CON XELATEX PUEDE FUNCIONAR CREO, PERO CON PDFlatex no. PDFlatex no entiende la fuente (fira) y utiliza las suyas}}
			\note[item]{\textcolor{red}{CUIDADO, SE ME PONE EL TEXTO DE COLOR BLANCO Y NO SE VE, ES CULPA DE USAR NOTAS (PFGPAGES) Y XELATEX}}
            \note[item]{\textcolor{red}{SI USO TODOs SE JOROBA LA FUENTE Y DA ERRORES. POR ESO HE PUESTO EL TEXTO QUE NO ES DE NOTAS SINO DE COSAS PARA CAMBIAR DE LAS DIAPOSITIVAS EN ROJO}}
            \note[item]{\textcolor{red}{ECHAR UN OJO A LA PORTADA, POR SI NO PRESENTARÁ EN JUNIO (Pone Junio, 2018)}}
\end{frame}

%SI NO ME GUSTA COMO PONGO LA PORTADA (begin frame... titlepage...) la puedo poner como la tenía antes, \maketitle sólo, sin begin frame ni nada. Está puesto con Begin frame porque así puedo añadir notas en esa diapo.
%\maketitle
\begin{frame}{Índice}
	\setbeamertemplate{section in toc}[sections numbered]
	\tableofcontents[hideallsubsections]
    \note{\LARGE \vfill
		\begin{center}
			\begin{enumerate}
				\item Son los capítulos del trabajo, el índice hace de ''Estructura de proyecto''.
				\vfill
			\end{enumerate}
		\end{center}}
\end{frame}
%%%%%%%%%%%%%%%%%%%%%%%%%%%%%%%%%%%%%%%%%%%%%%%%%%%%%%%%%
%%%%%%%%%%%%%%%%%%
%% INTRODUCCIÓN %%
%%%%%%%%%%%%%%%%%%
\section{Introducción}


\begin{frame}{Motivación y objetivos I}
\vspace{-2em}
Las redes definidas por software (SDN) surgen a principios de 2010 \alert{por necesidad}:
\begin{itemize}
	\item La mayoría de las redes tradicionales fueron diseñadas para aplicaciones cliente-servidor que se ejecutan en una infraestructura no virtualizada.
\end{itemize}

SDN se ha establecido en la adultez temprana como un producto conocido.

Es una realidad que muchas de las empresas y proveedores de servicios de todo el mundo ya han adoptado.


\note{\large \vfill
	\begin{center}
	\begin{enumerate}
		\item NO son la solución a un problema sin resolver sino que resuelven de una forma mas eficiente que las soluciones tradicionales. 		
		\vspace{2em}
		\item SDN ha crecido más allá de su adolescencia y euforia prematura...	
		\vspace{2em}	
		\item NO es una próxima novedad en el horizonte de la creación de redes...	
		\vspace{2em}
		\vfill
	\end{enumerate}
	\end{center}}
\end{frame}


\begin{frame}{Motivación y objetivos II}
\vspace{-2em}
\textit{Rohit Mehra} y \textit{Brad Casemore} en su previsión sobre SDN publicada en 2016:
\vspace{1.3em}

%\textit{\say{Virtualization, cloud, mobility, and now the Internet of Things (IoT) have exposed the limitations of traditional network architectures and operational models.}}

\begin{shadequote}{}
\large\textit{Virtualization, cloud, mobility, and now the Internet of Things (IoT) have exposed the limitations of traditional network architectures and operational models.}
\end{shadequote}

%\begin{shadequote}[r]{Douglas Adams}
%A common mistake that people make when trying to design something completely foolproof is to underestimate the ingenuity of complete fools.
%\end{shadequote}

%\begin{shadequote}[c]{Douglas Adams}
%A common mistake that people make when trying to design something completely foolproof is to underestimate the ingenuity of complete fools.
%\end{shadequote}

%\begin{shadequote}{}
%A common mistake that people make when trying to design something completely foolproof is to underestimate the ingenuity of complete fools.
%\end{shadequote}

\note[item]{\textcolor{red}{VER SI AÑADO ESTA DIAPO O NO, DECIDIRLO CUANDO TENGA TODAS LAS DIAPOS HECHAS, PARA VER SI SON DEMASIADAS. OCUPARÍA 15 SEGUNDOS EXPLICAR ESTA DIAPO, POR TIEMPO NO HABRÍA PROBLEMA}}
\end{frame}


% DIAPOSITIVA DEL HISTOGRAMA SOBRE LA ADOPCIÓN DE SDN
\begin{frame}[fragile]{Motivación y objetivos III}
\begin{figure}
\centering
\pgfplotsset{
	select row/.style={
		x filter/.code={\ifnum\coordindex=#1\else\def\pgfmathresult{}\fi}
	}
}

\pgfplotstableread[header=false]{
	74 {Suma de las que emplean o emplearán}
	26 {No}
	25 {En un futuro, sin fecha}
	21 {En los próximos 2 años}
	28 {Sí}
}\datatable

\hspace*{-6em}
\begin{tikzpicture}[yscale=0.7,xscale=0.7] %TAMAÑO DE LA GRÁFICA
\begin{axis}[
title= Adopción de las redes SDN en las empresas IT en 2016.,
xbar, bar shift=0pt,
enlarge y limits=0.2,
xmin=0,
ytick={0,...,5},
yticklabel style={text width=3cm,align=right},
yticklabels from table={\datatable}{1},
yticklabel style={align=right},
% xmajorgrids = true, %CON ESTO DESCOMENTADO MUESTRA UN GRID, SINO SOLO LAS BARRAS, SIN GRID.
bar width=8mm, 
width=12cm, height=8.5cm,  %CAMBIAR TAMAÑO DE LA GRAFICA, HORIZONTAL Y VERTICAL, HEIGHT PARA CAMBIAR EL  ESPACIO ENTRE BARRAS
xlabel={\% de las empresas IT encuestadas},
nodes near coords={\pgfmathprintnumber\pgfplotspointmeta\%},
nodes near coords align={horizontal}, 
]

\pgfplotsinvokeforeach{0,...,5}{
	\addplot table [select row=#1, y expr=#1] {\datatable};
}
\end{axis}
\end{tikzpicture}

\vspace{1.5em}\source{\href{https://www.channelinsider.com/networking/slideshows/enterprise-interest-in-sdn-adoption-picks-up-steam.html}{Channel Insider Networking - Michael Vizard}}

\note{\large \vfill
	\begin{center}
		\begin{enumerate}
			\item Debido a la creciente demanda en las redes, en estos años se ha visto una evolución en el mercado de SDN.	
			\vspace{2em}	
			\item Es por esto y por el TFG DE RUBEN que surge el proyecto.	
			\vspace{2em}
			\vfill
		\end{enumerate}
\end{center}}

\end{figure}
\end{frame}





\begin{frame}{Motivación y objetivos IV}
\vspace*{-2em}
\begin{alertblock}{\LARGE\textbf{OBJETIVOS}}
\begin{itemize}
	\vspace{1.2em}
	\item[\textbf{>>}] Exponer dos casos de uso real de las redes SDN.
	\vspace{1.3em}
    \item[\textbf{>>>>}] Aplicar técnicas de ingeniería de tráfico en dichos casos.
    \vspace{1.3em}
	\item[\textbf{>>>>>>}] Implementarlos en Mininet haciendo uso del controlador \texttt{Ryu} y un script en \texttt{Python}.
\end{itemize}
\end{alertblock}
\note[item]{\textcolor{blue}{COMPROBAR QUE LOS TITULOS DE ESTAS DIAPOS ESTÁN BIEN: I, II, III, IV, V, ...}}
\end{frame}

\section{Ámbito de trabajo}

\begin{frame}{Estado del arte}
\vspace{2em}
\begin{figure}[h!]
\centering
\hspace*{-0.30in}% PARA AJUSTAR IZQ-DCH LA FIGURA.
\tikzset{every picture/.style={line width=0.5pt}} %set default line width to 0.75pt      
\resizebox{0.9\textwidth}{!}{  % ESCALAR LA FIGURA, CON ESTE COMANDO SE QUE NO SE ME VA A SALIR FUERA DE TEXTWIDTH
\begin{tikzpicture}[x=0.75pt,y=0.75pt,yscale=-1,xscale=1]
%uncomment if require: \path (0,379); %set diagram left start at 0, and has height of 379

\draw [color={rgb, 255:red, 0; green, 0; blue, 0 }  ,draw opacity=1 ][line width=0.5]  [dash pattern={on 4.5pt off 4.5pt}]  (232.61,255.91) -- (595.88,255.91) ;


\draw [color={rgb, 255:red, 0; green, 0; blue, 0 }  ,draw opacity=1 ][line width=0.5]  [dash pattern={on 4.5pt off 4.5pt}]  (93.66,347) -- (456.93,347) ;


\draw [color={rgb, 255:red, 0; green, 0; blue, 0 }  ,draw opacity=1 ][line width=0.5]  [dash pattern={on 4.5pt off 4.5pt}]  (93.66,347) -- (232.61,255.91) ;


\draw [color={rgb, 255:red, 0; green, 0; blue, 0 }  ,draw opacity=1 ][line width=0.5]  [dash pattern={on 4.5pt off 4.5pt}]  (456.93,347) -- (595.88,255.91) ;


\draw  [color={rgb, 255:red, 0; green, 0; blue, 0 }  ,draw opacity=1 ][line width=0.5]   (232.42, 268.26) rectangle (306.49, 279.59)   ;
\draw  [color={rgb, 255:red, 0; green, 0; blue, 0 }  ,draw opacity=1 ][line width=0.5]   (232.42, 279.59) rectangle (306.49, 290.91)   ;
\draw  [color={rgb, 255:red, 0; green, 0; blue, 0 }  ,draw opacity=1 ][line width=0.5]   (232.42, 313.36) rectangle (306.49, 324.69)   ;
\draw  [color={rgb, 255:red, 0; green, 0; blue, 0 }  ,draw opacity=1 ][line width=0.5]   (232.42, 324.69) rectangle (306.49, 336.01)   ;
\draw  [color={rgb, 255:red, 0; green, 0; blue, 0 }  ,draw opacity=1 ][line width=0.5]   (382.09, 267.7) rectangle (456.17, 279.02)   ;
\draw  [color={rgb, 255:red, 0; green, 0; blue, 0 }  ,draw opacity=1 ][line width=0.5]   (382.09, 279.02) rectangle (456.17, 290.35)   ;
\draw  [color={rgb, 255:red, 0; green, 0; blue, 0 }  ,draw opacity=1 ][line width=0.5]   (382.09, 313.36) rectangle (456.17, 324.69)   ;
\draw  [color={rgb, 255:red, 0; green, 0; blue, 0 }  ,draw opacity=1 ][line width=0.5]   (382.09, 324.69) rectangle (456.17, 336.01)   ;
\draw [color={rgb, 255:red, 0; green, 0; blue, 0 }  ,draw opacity=1 ][line width=0.5]    (306.49,279.59) -- (382.09,279.02) ;


\draw [color={rgb, 255:red, 0; green, 0; blue, 0 }  ,draw opacity=1 ][line width=0.5]    (306.49,324.69) -- (382.09,324.69) ;


\draw [color={rgb, 255:red, 0; green, 0; blue, 0 }  ,draw opacity=1 ][line width=0.5]    (419.13,313.36) -- (419.13,290.35) ;


\draw [color={rgb, 255:red, 0; green, 0; blue, 0 }  ,draw opacity=1 ][line width=0.5]    (269.46,313.36) -- (269.46,290.91) ;


\draw [color={rgb, 255:red, 0; green, 0; blue, 0 }  ,draw opacity=1 ][line width=0.5]  [dash pattern={on 4.5pt off 4.5pt}]  (232.59,134.05) -- (595.86,134.05) ;


\draw [color={rgb, 255:red, 0; green, 0; blue, 0 }  ,draw opacity=1 ][line width=0.5]  [dash pattern={on 4.5pt off 4.5pt}]  (93.65,225.14) -- (456.92,225.14) ;


\draw [color={rgb, 255:red, 0; green, 0; blue, 0 }  ,draw opacity=1 ][line width=0.5]  [dash pattern={on 4.5pt off 4.5pt}]  (93.65,225.14) -- (232.59,134.05) ;


\draw [color={rgb, 255:red, 0; green, 0; blue, 0 }  ,draw opacity=1 ][line width=0.5]  [dash pattern={on 4.5pt off 4.5pt}]  (456.92,225.14) -- (595.86,134.05) ;


\draw [color={rgb, 255:red, 0; green, 0; blue, 0 }  ,draw opacity=1 ][line width=0.5]  [dash pattern={on 4.5pt off 4.5pt}]  (232.61,16.59) -- (595.88,16.59) ;


\draw [color={rgb, 255:red, 0; green, 0; blue, 0 }  ,draw opacity=1 ][line width=0.5]  [dash pattern={on 4.5pt off 4.5pt}]  (93.66,107.69) -- (456.93,107.69) ;


\draw [color={rgb, 255:red, 0; green, 0; blue, 0 }  ,draw opacity=1 ][line width=0.5]  [dash pattern={on 4.5pt off 4.5pt}]  (93.66,107.69) -- (232.61,16.59) ;


\draw [color={rgb, 255:red, 0; green, 0; blue, 0 }  ,draw opacity=1 ][line width=0.5]  [dash pattern={on 4.5pt off 4.5pt}]  (456.93,107.69) -- (595.88,16.59) ;


\draw  [color={rgb, 255:red, 0; green, 0; blue, 0 }  ,draw opacity=1 ][line width=0.5]   (243.67, 47.04) rectangle (300.84, 70.14)   ;
\draw  [color={rgb, 255:red, 0; green, 0; blue, 0 }  ,draw opacity=1 ][line width=0.5]   (321.25, 47.04) rectangle (378.41, 70.14)   ;
\draw  [color={rgb, 255:red, 0; green, 0; blue, 0 }  ,draw opacity=1 ][line width=0.5]   (398.01, 47.04) rectangle (455.18, 70.14)   ;
\draw    (124, 72.67) rectangle (124, 72.67)   ;
\draw  [color={rgb, 255:red, 0; green, 0; blue, 0 }  ,draw opacity=1 ][line width=0.5]   (280, 161.3) rectangle (420, 187.14)   ;
\draw    (186,320) .. controls (247,268) and (70.67,336) .. (116,280) ;
\draw [shift={(116,280)}, rotate = 486.99] [color={rgb, 255:red, 0; green, 0; blue, 0 }  ]   (0,0) .. controls (3.31,-0.3) and (6.95,-1.4) .. (10.93,-3.29)(0,0) .. controls (3.31,0.3) and (6.95,1.4) .. (10.93,3.29)   ;
\draw [shift={(186,320)}, rotate = 318.36] [color={rgb, 255:red, 0; green, 0; blue, 0 }  ]   (0,0) .. controls (3.31,-0.3) and (6.95,-1.4) .. (10.93,-3.29)(0,0) .. controls (3.31,0.3) and (6.95,1.4) .. (10.93,3.29)   ;
\draw    (136,260) .. controls (176,230) and (106,240) .. (146,210) ;
\draw [shift={(146,210)}, rotate = 501.69] [color={rgb, 255:red, 0; green, 0; blue, 0 }  ]   (0,0) .. controls (3.31,-0.3) and (6.95,-1.4) .. (10.93,-3.29)(0,0) .. controls (3.31,0.3) and (6.95,1.4) .. (10.93,3.29)   ;
\draw [shift={(136,260)}, rotate = 321.47] [color={rgb, 255:red, 0; green, 0; blue, 0 }  ]   (0,0) .. controls (3.31,-0.3) and (6.95,-1.4) .. (10.93,-3.29)(0,0) .. controls (3.31,0.3) and (6.95,1.4) .. (10.93,3.29)   ;
\draw    (146,139) .. controls (186,109) and (116,119) .. (156,89) ;
\draw [shift={(156,89)}, rotate = 501.69] [color={rgb, 255:red, 0; green, 0; blue, 0 }  ]   (0,0) .. controls (3.31,-0.3) and (6.95,-1.4) .. (10.93,-3.29)(0,0) .. controls (3.31,0.3) and (6.95,1.4) .. (10.93,3.29)   ;
\draw [shift={(146,139)}, rotate = 321.47] [color={rgb, 255:red, 0; green, 0; blue, 0 }  ]   (0,0) .. controls (3.31,-0.3) and (6.95,-1.4) .. (10.93,-3.29)(0,0) .. controls (3.31,0.3) and (6.95,1.4) .. (10.93,3.29)   ;
\draw    (200,201) .. controls (261,149) and (84.67,217) .. (130,161) ;
\draw [shift={(130,161)}, rotate = 486.99] [color={rgb, 255:red, 0; green, 0; blue, 0 }  ]   (0,0) .. controls (3.31,-0.3) and (6.95,-1.4) .. (10.93,-3.29)(0,0) .. controls (3.31,0.3) and (6.95,1.4) .. (10.93,3.29)   ;
\draw [shift={(200,201)}, rotate = 318.36] [color={rgb, 255:red, 0; green, 0; blue, 0 }  ]   (0,0) .. controls (3.31,-0.3) and (6.95,-1.4) .. (10.93,-3.29)(0,0) .. controls (3.31,0.3) and (6.95,1.4) .. (10.93,3.29)   ;
\draw    (66, 139.59) rectangle (186, 160)   ;
\draw    (46, 260) rectangle (186, 280)   ;

\draw (269.15,273.46) node [scale=0.7] [align=left] {Flow de datos};
\draw (270.07,285.16) node [scale=0.7] [align=left] {Forwarding};
\draw (269.15,318.56) node [scale=0.7] [align=left] {Flow de datos};
\draw (270.07,330.26) node [scale=0.7] [align=left] {Forwarding};
\draw (418.82,272.9) node [scale=0.7] [align=left] {Flow de datos};
\draw (419.74,284.59) node [scale=0.7] [align=left] {Forwarding};
\draw (418.82,318.56) node [scale=0.7] [align=left] {Flow de datos};
\draw (419.74,330.26) node [scale=0.7] [align=left] {Forwarding};
\draw (349.82,174.89) node [scale=1.44] [align=left] {Controlador};
\draw (273.89,58.59) node  [align=left] {App};
\draw (351.47,58.59) node  [align=left] {App};
\draw (428.23,58.59) node  [align=left] {App};
\draw (561.44,257.97) node  [align=left] {\textbf{Plano de datos}\\};
\draw (556.44,141.17) node  [align=left] {\textbf{Plano de control}\\\\};
\draw (553.11,26.3) node  [align=left] {\textbf{Plano de aplicación}\\\\};
\draw (126,150) node  [align=left] {NortBound API};
\draw (116,270) node  [align=left] {SouthBound API};
\end{tikzpicture}
}
\caption{Arquitectura de alto nivel SDN.}
\label{fig: capas_sdn}
\end{figure}
\note{\tiny \vfill
	\begin{center}
		\begin{enumerate}
        \item La Open Networking Foundation define una arquitectura de alto nivel para SDN con tres capas o planos principales, como se muestra en la figura.
			\vspace{1.5em}	
			\item SDN separa Plano de Control - Plano de Datos
			\vspace{1.5em}	
			\item Simplifica la operación en el plano de datos -> dispositivos de red menos costosos.
			\vspace{1.5em}
            
			\item Centraliza el control (toma de decisiones) en la red.
            %centraliza el estado de la red y la toma de deqcisiones
			\vspace{1.5em}	
            
			\item Programabilidad de la red, administración simplificada y autónoma.
			\vspace{1.5em}	
            
			\item Estimula la aplicacion \Rightarrow abre mercados y oportunidades para todo el sector.
			\vspace{1.5em}	
            
			\item \textcolor{blue}{COMPROBAR SI EXISTEN MÁS DIAPOS CON EL MISMO TITULO PARA PONER: I, II, III, IV, V, ...}
			\vspace{1.5em}	
			\vfill
		\end{enumerate}
        \end{center}}
% \note[item]{SDN separa Plano de Control - Plano de Datos}
% \note[item]{Simplifica la operación en el plano de datos -> dispositivos de red menos costosos.}
% \note[item]{centraliza el ''estado'' de la red y la toma de decisiones.}
% \note[item]{Programabilidad de la red, administración simplificada y autónoma.}
% \note[item]{Estimula la aplicacion \Rightarrow abre mercados y oportunidades para todo el sector.}
% \note[item]{\textcolor{blue}{COMPROBAR SI EXISTEN MÁS DIAPOS CON EL MISMO TITULO PARA PONER: I, II, III, IV, V, ...}}
\end{frame}


\begin{frame}{Protocolo OpenFlow I}
\vspace*{-2em}
\begin{alertblock}{\LARGE\textbf{OpenFlow}}
\vspace{1.2em}
Protocolo estandarizado por Open Networking Foundation en
2013 que define la comunicación hacia el sur (Southbound) entre un controlador y un switch OpenFlow.
\end{alertblock}

El tráfico se clasifica en flows en función de sus características.  


\note[item]{}
\note[item]{}
\note[item]{}
\note[item]{\textcolor{blue}{COMPROBAR QUE LOS TITULOS DE ESTAS DIAPOS ESTÁN BIEN: I, II, III, IV, V, ...}}
\end{frame}


\begin{frame}{Protocolo OpenFlow II}
\vspace*{0em}
	\begin{figure}[h!]
		\centering
		 \hspace*{-0.46in}% PARA AJUSTAR IZQ-DCH LA FIGURA.
	\tikzset{every picture/.style={line width=0.75pt}} %set default line width to 0.75pt      
	\resizebox{0.9\textwidth}{!}{  % ESCALAR LA FIGURA, CON ESTE COMANDO SE QUE NO SE ME VA A SALIR FUERA DE TEXTWIDTH
	\begin{tikzpicture}[x=0.75pt,y=0.75pt,yscale=-1,xscale=1]
	%uncomment if require: \path (0,409); %set diagram left start at 0, and has height of 409
	
	\draw    (214.62, 276.67) rectangle (214.62, 276.67)   ;
	\draw [rotate around= { 314.97: (464.79, 240.36)
	}] [fill={rgb, 255:red, 155; green, 155; blue, 155 }  ,fill opacity=0.56 ]  (435.74, 211.3) rectangle (493.84, 269.41)   ;
	\draw    (395.7,240) -- (424.15,240.38) ;
	\draw [shift={(424.15,240.38)}, rotate = 180.76] [color={rgb, 255:red, 0; green, 0; blue, 0 }  ]   (0,0) .. controls (3.31,-0.3) and (6.95,-1.4) .. (10.93,-3.29)(0,0) .. controls (3.31,0.3) and (6.95,1.4) .. (10.93,3.29)   ;
	
	\draw    (554.95,285) -- (607.17,285.33) ;
	\draw [shift={(607.17,285.33)}, rotate = 180.37] [color={rgb, 255:red, 0; green, 0; blue, 0 }  ]   (0,0) .. controls (3.31,-0.3) and (6.95,-1.4) .. (10.93,-3.29)(0,0) .. controls (3.31,0.3) and (6.95,1.4) .. (10.93,3.29)   ;
	
	\draw    (505.43,240.34) -- (554.2,239.67) ;
	
	
	\draw    (554.2,239.67) -- (554.95,285) ;
	
	
	\draw    (523.7,239.5) -- (523.26,50.07) ;
	\draw [shift={(523.26,50.07)}, rotate = 449.87] [color={rgb, 255:red, 0; green, 0; blue, 0 }  ]   (0,0) .. controls (3.31,-0.3) and (6.95,-1.4) .. (10.93,-3.29)(0,0) .. controls (3.31,0.3) and (6.95,1.4) .. (10.93,3.29)   ;
	
	\draw  [color={rgb, 255:red, 245; green, 166; blue, 35 }  ,draw opacity=0.65 ][fill={rgb, 255:red, 245; green, 166; blue, 35 }  ,fill opacity=0.54 ]  (161.27, 225) rectangle (395.17, 255)   ;
	\draw    (472.43, 144.67) rectangle (472.43, 144.67)   ;
	\draw  [color={rgb, 255:red, 80; green, 227; blue, 194 }  ,draw opacity=1 ][fill={rgb, 255:red, 80; green, 227; blue, 194 }  ,fill opacity=0.64 ]  (405.17, 10.67) rectangle (650.17, 49.76)   ;
	\draw    (175.17, 121.33) rectangle (422.33, 199)   ;
	\draw    (281.2,121.33) -- (281.2,199.15) ;
	
	
	\draw    (358.41,121.19) -- (358.41,199) ;
	
	
	\draw    (175.17,147.51) -- (423.17,146.94) ;
	
	
	\draw    (175.86,173.4) -- (422.33,173.4) ;
	
	
	\draw    (110.44,240.33) -- (160.38,240) ;
	\draw [shift={(160.38,240)}, rotate = 539.62] [color={rgb, 255:red, 0; green, 0; blue, 0 }  ]   (0,0) .. controls (3.31,-0.3) and (6.95,-1.4) .. (10.93,-3.29)(0,0) .. controls (3.31,0.3) and (6.95,1.4) .. (10.93,3.29)   ;
	
	\draw    (464.81,281.45) -- (464.17,325.67) ;
	\draw [shift={(464.17,325.67)}, rotate = 270.84000000000003] [color={rgb, 255:red, 0; green, 0; blue, 0 }  ]   (0,0) .. controls (3.31,-0.3) and (6.95,-1.4) .. (10.93,-3.29)(0,0) .. controls (3.31,0.3) and (6.95,1.4) .. (10.93,3.29)   ;
	
	\draw  [dash pattern={on 1.69pt off 2.76pt}][line width=1.5]   (56.42, 113.33) rectangle (663.33, 335.67)   ;
	\draw    (251,199) -- (251.17,224.33) ;
	\draw [shift={(251.17,224.33)}, rotate = 269.62] [color={rgb, 255:red, 0; green, 0; blue, 0 }  ]   (0,0) .. controls (3.31,-0.67) and (6.95,-2.3) .. (10.93,-4.9)(0,0) .. controls (3.31,0.67) and (6.95,2.3) .. (10.93,4.9)   ;
	
	\draw  [color={rgb, 255:red, 0; green, 0; blue, 0 }  ,draw opacity=1 ][dash pattern={on 5.63pt off 4.5pt}][line width=1.5]   (520.08, 84.83) circle [x radius= 36.92, y radius= 12.83]  ;
	
	\draw     (63.55, 180.85) rectangle (108.88, 199.15)   ;
	\draw (86.22,190) node [scale=0.9] [align=left] { \ Port 1 };
	\draw     (64.45, 230.85) rectangle (109.78, 249.15)   ;
	\draw (87.11,240) node [scale=0.9] [align=left] { \ Port 2 };
	\draw     (63.55, 280.85) rectangle (108.88, 299.15)   ;
	\draw (86.22,290) node [scale=0.9] [align=left] { \ Port n };
	\draw (279.33,240) node [color={rgb, 255:red, 0; green, 0; blue, 0 }  ,opacity=1 ] [align=left] {Función de Matching de paquetes};
	\draw (467.82,240) node  [align=left] {\textbf{Match?}};
	\draw     (608.07, 175.85) rectangle (653.4, 194.15)   ;
	\draw (630.74,185) node [scale=0.9] [align=left] { \ Port 1 };
	\draw     (608.07, 225.85) rectangle (653.4, 244.15)   ;
	\draw (630.74,235) node [scale=0.9] [align=left] { \ Port 2 };
	\draw     (608.07, 275.85) rectangle (653.4, 294.15)   ;
	\draw (630.74,285) node [scale=0.9] [align=left] { \ Port n };
	\draw (521.47,30.22) node [color={rgb, 255:red, 0; green, 0; blue, 0 }  ,opacity=1 ] [align=left] {\textbf{Controlador OpenFlow}};
	\draw (225.16,134.14) node [scale=0.7] [align=left] {\textbf{Cabecera}};
	\draw (320.22,133.28) node [scale=0.7] [align=left] {\textbf{Contadores}};
	\draw (390.79,133.28) node [scale=0.7] [align=left] {\textbf{Acciones}};
	\draw (225.99,160.59) node [scale=0.7] [align=left] {xxxxxxxxxxx};
	\draw (225.99,183.64) node [scale=0.7] [align=left] {xxxxxxxxxxx};
	\draw (321.05,159.74) node [scale=0.7] [align=left] {yyyyyyyyy};
	\draw (321.05,183.64) node [scale=0.7] [align=left] {yyyyyyyyy};
	\draw (388.3,186.2) node [scale=0.7] [align=left] {A, B, C};
	\draw (389.13,159.74) node [scale=0.7] [align=left] {A, B, C};
	\draw  [color={rgb, 255:red, 126; green, 211; blue, 33 }  ,draw opacity=1 ][fill={rgb, 255:red, 126; green, 211; blue, 33 }  ,fill opacity=0.56 ]  (535, 274) circle [x radius= 14.3, y radius= 14.3]   ;
	\draw (535,274) node [scale=1.44] [align=left] {A};
	\draw (548,299) node [scale=0.9,color={rgb, 255:red, 126; green, 211; blue, 33 }  ,opacity=1 ] [align=left] {para el Outport};
	\draw  [color={rgb, 255:red, 208; green, 2; blue, 27 }  ,draw opacity=1 ][fill={rgb, 255:red, 208; green, 2; blue, 27 }  ,fill opacity=0.68 ]  (436, 296) circle [x radius= 14.92, y radius= 14.92]   ;
	\draw (436,296) node [scale=1.44] [align=left] {C};
	\draw (350,299) node [scale=0.9,color={rgb, 255:red, 208; green, 2; blue, 27 }  ,opacity=1 ] [align=left] {Notificar Controlador\\o descartar paquete};
	\draw  [color={rgb, 255:red, 248; green, 231; blue, 28 }  ,draw opacity=1 ][fill={rgb, 255:red, 248; green, 231; blue, 28 }  ,fill opacity=0.52 ]  (547, 176) circle [x radius= 14.3, y radius= 14.3]   ;
	\draw (547,176) node [scale=1.44] [align=left] {B};
	\draw (587,147) node [scale=0.9,color={rgb, 255:red, 214; green, 199; blue, 0 }  ,opacity=1 ] [align=left] {Para el controlador};
	\draw (139,98) node  [align=left] {\textbf{Switch OpenFlow}};
	\draw (402,84) node [color={rgb, 255:red, 0; green, 0; blue, 0 }  ,opacity=1 ] [align=left] {\textbf{SouthBound API}};
	\end{tikzpicture}
	}	
	\caption{Switch OpenFlow. Operación básica.}
	\label{fig: openflow_basic}
\end{figure}
\note[item]{\textcolor{blue}{COMPROBAR QUE LOS TITULOS DE ESTAS DIAPOS ESTÁN BIEN: I, II, III, IV, V, ...}}
\end{frame}



\begin{frame}{Elementos}
\begin{alertblock}{\LARGE}
\begin{itemize}
\large
	\item \textbf{Ryu}
    \vspace*{1em}
	\item \textbf{VirtualBox}
	\vspace*{1em}
	\item \textbf{Mininet}
	\vspace*{1em}
    \item \textbf{iPerf}
\end{itemize}
\end{alertblock}
\note[item]{Contar para que se he usado cada uno.}
\note[item]{\textcolor{blue}{COMPROBAR QUE LOS TITULOS DE ESTAS DIAPOS ESTÁN BIEN: I, II, III, IV, V, ...}}
\end{frame}
\begin{frame}
\section{Definición del escenario de aplicación}

\note{\large \vfill
	\begin{center}
		\begin{enumerate}
			\item Se definen los escenarios de aplicación que existen y cómo se pueden mejorar utilizando SDN. 	
			\vspace{2em}	
			\item 2 casos de uso.	
			\vspace{2em}
			\item Se explica Ingeniería de tráfico y QoS.	
			\vspace{2em}
			\vfill
		\end{enumerate}
\end{center}}
\end{frame}

\begin{frame}{Ingeniería de Tráfico}
\vspace*{-2em}
\begin{alertblock}{\LARGE\textbf{Ingeniería de Tráfico}}
\vspace{1.2em}
\large Es una aplicación de red importante que estudia
la medición y gestión del tráfico.

Diseña mecanismos de enrutamiento para guiar el tráfico de red a fin de mejorar la utilización de los recursos y cumplir mejor los requisitos de QoS.
\end{alertblock}
\note[item]{\textcolor{red}{ESTAS NOTAS YA SE HABRÁN CONTANDO EN DIAPOS ANTERIORES, EN LO QUE ES SDN. PERO BUENO, NO PASA NADA POR REPETIR QUE ESAS CARACTERISTICAS QUE TIENE SDN VAN BIEN PARA LA ING. TRÁFICO.}
\note[item{En comparación con las redes tradicionales, SDN tiene muchas ventajas para ser
compatible con TE debido a sus características distintivas, como el aislamiento de
los planos de control y datos, el control centralizado global y la programabilidad del
comportamiento de la red.}
\end{frame}



\begin{frame}{Calidad de servicio - QoS}
\vspace*{-2em}
\begin{alertblock}
{\LARGE\textbf{AAAAAAAAAA}}
\vspace{1.2em}
AAAAAAAAAAAAAAAAAAAAAAAAAAAAAAAAAAAAAAAAAAAAAAAAAAAAAAAAAAAAAAAAAA
\end{alertblock}
\note[item]{\textcolor{blue}{COMPROBAR QUE LOS TITULOS DE ESTAS DIAPOS ESTÁN BIEN: I, II, III, IV, V, ...}}
\end{frame}



\begin{frame}{AAAAAAAAAAAAAAAAAAAAAAAAA}
\vspace*{-2em}
\begin{alertblock}
{\LARGE\textbf{AAAAAAAAAA}}
\vspace{1.2em}
AAAAAAAAAAAAAAAAAAAAAAAAAAAAAAAAAAAAAAAAAAAAAAAAAAAAAAAAAAAAAAAAAA
\end{alertblock}
\note[item]{\textcolor{blue}{COMPROBAR QUE LOS TITULOS DE ESTAS DIAPOS ESTÁN BIEN: I, II, III, IV, V, ...}}
\end{frame}


\section{Routing multicamino con balanceador de carga}
\section{Implementación}
\section{Conclusiones y líneas futuras}







\begin{frame}[fragile]{Metropolis}

The \themename theme is a Beamer theme with minimal visual noise
inspired by the \href{https://github.com/hsrmbeamertheme/hsrmbeamertheme}{\textsc{hsrm} Beamer
Theme} by Benjamin Weiss.

Enable the theme by loading

\begin{verbatim}    \documentclass{beamer}
\usetheme{metropolis}\end{verbatim}

Note, that you have to have Mozilla's \emph{Fira Sans} font and XeTeX
installed to enjoy this wonderful typography.
\end{frame}
\begin{frame}[fragile]{Sections}
Sections group slides of the same topic

\begin{verbatim}    \section{Elements}\end{verbatim}

for which \themename provides a nice progress indicator \ldots
\end{frame}


\begin{frame}{Metropolis title formats}
\themename supports 4 different title formats:
\begin{itemize}
\item Regular
\item \textsc{Small caps}
\item \textsc{all small caps}
\item ALL CAPS
\end{itemize}
They can either be set at once for every title type or individually.
\end{frame}

{
\metroset{titleformat frame=smallcaps}
\begin{frame}{Small caps}
This frame uses the \texttt{smallcaps} title format.

\begin{alertblock}{Potential Problems}
Be aware that not every font supports small caps. If for example you typeset your presentation with pdfTeX and the Computer Modern Sans Serif font, every text in small caps will be typeset with the Computer Modern Serif font instead.
\end{alertblock}
\end{frame}
}

{
\metroset{titleformat frame=allsmallcaps}
\begin{frame}{All small caps}
This frame uses the \texttt{allsmallcaps} title format.

\begin{alertblock}{Potential problems}
As this title format also uses small caps you face the same problems as with the \texttt{smallcaps} title format. Additionally this format can cause some other problems. Please refer to the documentation if you consider using it.

As a rule of thumb: just use it for plaintext-only titles.
\end{alertblock}
\end{frame}
}

{
\metroset{titleformat frame=allcaps}
\begin{frame}{All caps}
This frame uses the \texttt{allcaps} title format.

\begin{alertblock}{Potential Problems}
This title format is not as problematic as the \texttt{allsmallcaps} format, but basically suffers from the same deficiencies. So please have a look at the documentation if you want to use it.
\end{alertblock}
\end{frame}
}


\begin{frame}[fragile]{Typography}
\begin{verbatim}The theme provides sensible defaults to
\emph{emphasize} text, \alert{accent} parts
or show \textbf{bold} results.\end{verbatim}

\begin{center}becomes\end{center}

The theme provides sensible defaults to \emph{emphasize} text,
\alert{accent} parts or show \textbf{bold} results.
\end{frame}

\begin{frame}{Font feature test}
\begin{itemize}
\item Regular
\item \textit{Italic}
\item \textsc{Small Caps}
\item \textbf{Bold}
\item \textbf{\textit{Bold Italic}}
\item \textbf{\textsc{Bold Small Caps}}
\item \texttt{Monospace}
\item \texttt{\textit{Monospace Italic}}
\item \texttt{\textbf{Monospace Bold}}
\item \texttt{\textbf{\textit{Monospace Bold Italic}}}
\end{itemize}
\end{frame}

\begin{frame}{Lists}
\begin{columns}[T,onlytextwidth]
\column{0.33\textwidth}
Items
\begin{itemize}
\item Milk \item Eggs \item Potatoes
\end{itemize}

\column{0.33\textwidth}
Enumerations
\begin{enumerate}
\item First, \item Second and \item Last.
\end{enumerate}

\column{0.33\textwidth}
Descriptions
\begin{description}
\item[PowerPoint] Meeh. \item[Beamer] Yeeeha.
\end{description}
\end{columns}
\end{frame}
\begin{frame}{Animation}
\begin{itemize}[<+- | alert@+>]
\item \alert<4>{This is\only<4>{ really} important}
\item Now this
\item And now this
\end{itemize}
\end{frame}
\begin{frame}{Figures}
\begin{figure}
\newcounter{density}
\setcounter{density}{20}
\begin{tikzpicture}
\def\couleur{alerted text.fg}
\path[coordinate] (0,0)  coordinate(A)
++( 90:5cm) coordinate(B)
++(0:5cm) coordinate(C)
++(-90:5cm) coordinate(D);
\draw[fill=\couleur!\thedensity] (A) -- (B) -- (C) --(D) -- cycle;
\foreach \x in {1,...,40}{%
\pgfmathsetcounter{density}{\thedensity+20}
\setcounter{density}{\thedensity}
\path[coordinate] coordinate(X) at (A){};
\path[coordinate] (A) -- (B) coordinate[pos=.10](A)
-- (C) coordinate[pos=.10](B)
-- (D) coordinate[pos=.10](C)
-- (X) coordinate[pos=.10](D);
\draw[fill=\couleur!\thedensity] (A)--(B)--(C)-- (D) -- cycle;
}
\end{tikzpicture}
\caption{Rotated square from
\href{http://www.texample.net/tikz/examples/rotated-polygons/}{texample.net}.}
\end{figure}
\end{frame}
\begin{frame}{Tables}
\begin{table}
\caption{Largest cities in the world (source: Wikipedia)}
\begin{tabular}{@{} lr @{}}
\toprule
City & Population\\
\midrule
Mexico City & 20,116,842\\
Shanghai & 19,210,000\\
Peking & 15,796,450\\
Istanbul & 14,160,467\\
\bottomrule
\end{tabular}
\end{table}
\end{frame}
\begin{frame}{Blocks}
Three different block environments are pre-defined and may be styled with an
optional background color.

\begin{columns}[T,onlytextwidth]
\column{0.5\textwidth}
\begin{block}{Default}
Block content.
\end{block}

\begin{alertblock}{Alert}
Block content.
\end{alertblock}

\begin{exampleblock}{Example}
Block content.
\end{exampleblock}

\column{0.5\textwidth}

\metroset{block=fill}

\begin{block}{Default}
Block content.
\end{block}

\begin{alertblock}{Alert}
Block content.
\end{alertblock}

\begin{exampleblock}{Example}
Block content.
\end{exampleblock}

\end{columns}
\end{frame}
\begin{frame}{Math}
\begin{equation*}
e = \lim_{n\to \infty} \left(1 + \frac{1}{n}\right)^n
\end{equation*}
\end{frame}
\begin{frame}{Line plots}
\begin{figure}
\begin{tikzpicture}
\begin{axis}[
mlineplot,
width=0.9\textwidth,
height=6cm,
]

\addplot {sin(deg(x))};
\addplot+[samples=100] {sin(deg(2*x))};

\end{axis}
\end{tikzpicture}
\end{figure}
\end{frame}
\begin{frame}{Bar charts}
\begin{figure}
\begin{tikzpicture}
\begin{axis}[
mbarplot,
xlabel={Foo},
ylabel={Bar},
width=0.9\textwidth,
height=6cm,
]

\addplot plot coordinates {(1, 20) (2, 25) (3, 22.4) (4, 12.4)};
\addplot plot coordinates {(1, 18) (2, 24) (3, 23.5) (4, 13.2)};
\addplot plot coordinates {(1, 10) (2, 19) (3, 25) (4, 15.2)};

\legend{lorem, ipsum, dolor}

\end{axis}
\end{tikzpicture}
\end{figure}
\end{frame}
\begin{frame}{Quotes}
\begin{quote}
Veni, Vidi, Vici
\end{quote}
\end{frame}

{%
\setbeamertemplate{frame footer}{My custom footer }

\begin{frame}[fragile]{Frame footer}
\themename defines a custom beamer template to add a text to the footer. It can be set via
\begin{verbatim}\setbeamertemplate{frame footer}{My custom footer}\end{verbatim}
\end{frame}
}

\begin{frame}{References}
Some references to showcase [allowframebreaks] \cite{knuth92,ConcreteMath,Simpson,Er01,greenwade93}
\end{frame}

\begin{frame}{Summary}

Get the source of this theme and the demo presentation from

\begin{center}\url{github.com/matze/mtheme}\end{center}

The theme \emph{itself} is licensed under a
\href{http://creativecommons.org/licenses/by-sa/4.0/}{Creative Commons
Attribution-ShareAlike 4.0 International License}.

\begin{center}\ccbysa\end{center}

\end{frame}

\begin{frame}[standout]
Questions?
\end{frame}

\appendix

\begin{frame}[fragile]{Backup slides}
Sometimes, it is useful to add slides at the end of your presentation to
refer to during audience questions.

The best way to do this is to include the \verb|appendixnumberbeamer|
package in your preamble and call \verb|\appendix| before your backup slides.

\themename will automatically turn off slide numbering and progress bars for
slides in the appendix.
\end{frame}

\begin{frame}[allowframebreaks]{References}

\bibliography{demo}
\bibliographystyle{abbrv}

\end{frame}

\end{document}
