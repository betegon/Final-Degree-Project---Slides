\documentclass[10pt,spanish]{beamer}
%%documentclass[notes,spanish,10pt]{beamer}       % print frame + notes

% TILDES Y DEMÁS EN ESPAÑOL
\usepackage[spanish]{babel}
\usepackage[T1]{fontenc}
\usepackage[utf8]{inputenc}

\usetheme{metropolis}
\usepackage{appendixnumberbeamer}

\usepackage{booktabs}
\usepackage[scale=2]{ccicons}

\usepackage{pgfplots}
\usepgfplotslibrary{dateplot}
\usepackage{xspace}
\newcommand{\themename}{\textbf{\textsc{metropolis}}\xspace}
\setbeamertemplate{footline}[frame number]

% Comando para poder añadir la fuente en mis figuras
\newcommand{\source}[1]{{\textbf{Fuente}: {#1}} }

\usepackage{hyperref} % LINKS


\usepackage{pgfpages} %PARA LAS NOTAS
\setbeameroption{show notes}                        
\setbeameroption{show notes on second screen=right} %% en terminal, $: pdfpc main.pdf --notes=right
%https://bugs.kde.org/show_bug.cgi?id=152585
\setbeamertemplate{note page}{\pagecolor{yellow!5}\insertnote}\usepackage{palatino}


\usepackage{dirtytalk} % PARA CITAR, CITAS, QUOTES.

%TODOs notas:
\usepackage[colorinlistoftodos,prependcaption,textsize=small]{todonotes} %% PARA QUE NO APAREZCAN [disable]


%%%%%%%%%%%%%%%%%%%%%%%%%%%%%%%%%%%%%%%%%%%%%%%%%%%%%%%%%%%%%%%%%%%%%%%%%%%%%%%%
%%%%%%%%%%%%%%%%%%%%%%%%%%%%%%%%%%%%%%%%%%%%%%%%%%%%%%%%%%%%%%%%%%%%%%%%%%%%%%%%
%%%%%%%%%%%%%%%%%%%%%%%%%%%%%%%%%%%%%%%%%%%%%%%%%%%%%%%%%%%%%%%%%%%%%%%%%%%%%%%%
% DATOS DE LA PORTADA Y DEMÁS
%%%%%%%%%%%%%%%%%%%%%%%%%%%%%%%%%%%%%%%%%%%%%%%%%%%%%%%%%%%%%%%%%%%%%%%%%%%%%%%%
%%%%%%%%%%%%%%%%%%%%%%%%%%%%%%%%%%%%%%%%%%%%%%%%%%%%%%%%%%%%%%%%%%%%%%%%%%%%%%%%
%%%%%%%%%%%%%%%%%%%%%%%%%%%%%%%%%%%%%%%%%%%%%%%%%%%%%%%%%%%%%%%%%%%%%%%%%%%%%%%%


\title{Estudio de técnicas de Ingeniería de Tráfico basadas en SDN}
\subtitle{Study of SDN Traffic Engineering techniques}

%\author{Betegon Garcia, Miguel}
%\institute{Center for modern beamer themes}
\author[M. Betegón]{ Betegón García, Miguel\inst{1} \\ {\ttfamily miguel.betegon@alumnos.unican.es}} 

\institute[UC] % (optional)
{
	\inst{1}%
	Grado en Ingeniería de Tecnologías de Telecomunicación
}
%\date{\today}
\date{Junio, 2018}
\logo{
	\includegraphics[width=1.1cm,keepaspectratio]{figuras/git.png}
}
\titlegraphic{
	
	\includegraphics[width=3.5em]{figuras/uc.png}\hfill
	\includegraphics[width=5em]{figuras/uccc_no_letras.png}\hfill
	\includegraphics[width=3.2em]{figuras/git.png}
}

\setbeamertemplate{title page}{
	\begin{minipage}[b][\paperheight]{\textwidth}
		\vfill%
		\ifx\inserttitle\@empty\else\usebeamertemplate*{title}\fi
		\ifx\insertsubtitle\@empty\else\usebeamertemplate*{subtitle}\fi
		\usebeamertemplate*{title separator}
		\ifx\beamer@shortauthor\@empty\else\usebeamertemplate*{author}\fi
		\ifx\insertdate\@empty\else\usebeamertemplate*{date}\fi
		\ifx\insertinstitute\@empty\else\usebeamertemplate*{institute}\fi
		\vfill
		\ifx\inserttitlegraphic\@empty\else\inserttitlegraphic\fi
		\vspace*{1cm}
	\end{minipage}
}

%	\titlegraphic{\vspace{-1.6em}\includegraphics[height=1.5cm]{logo.pdf}}
\pgfplotsset{width=10cm,compat=1.9}
\usepackage{pgfplotstable}


%% ---- COMIENZA EL DOCUMENTO ----
%%%%%%%%%%%%%%%%%%%%%%%%%%%%%%%%%%%%%%%%%%%%%%%%%%%%%%%%%%%%%%%%%%%%%%%%%%%%%%%%%%%%%%%%%%%%%%%%%%%%%%%%%%%%%%%%%%%%%%%%%%%%%%%%%%%%%%%%%%%%%%%%%%%%%%%%%%%%%%%%%%%%%%%%%%%%%%%%%%%%%%%%%%%%%%%%%%%%%%%%%%%%%%%%%%%%%%%%%%%%%%%%%%%%%%%%%%%%%%%%%%%%%%%%%%%%%%%%%%%%%%%%%%%%%%%%%%%%%%%%%%%%%%%%%%%%%%%%%%%%%%%%%%%%%%%%%%%%%%%%%%%%%%%%%%%%%%%%%%%%%%%%%%%%%%%%%%%%%%%%%%%%%%%%%%%%%%%%%%%%%%%%%%%%%%%%%%%%%%%%%%%%%%%%%%%%%%%%%%%%%%%%%%%%%%%%%%%%%%%%%%%%%%%%%%%%%%%%%%%%%%%%%%%%%%%%%%%%%%%%%%%%%%%%%%%%%%%%%%%%%%%%%%%%%%%%%%%%%%%%%%%%%%%%%%%%%%%%%%%%%%%%%%%%%%%%%%%%%%%%%%%%%%%%%%%%%%%%%%%%%%%%%%%%%%%%%%%%%%%%%%%%%%%%%%%%%%%%%%%%%%%%%%%%%%%%%%%%%%%%%%%%%%%%%%%%%%%%%%%%%%%%%%%%%%%%%%%%%%%%%%%%%%%%%%%%%%%%%%%%%%%%%%%%%%%%%%%%%%%%%%%%%%%%%%%%%%%%%%%%%%%%%%%%%%%%%%%%%%%%%%%%%%%%%%%%%%%%%%%%%%%%%%%%%%%%%%%%%%%%%%%%%%
\begin{document}
	\maketitle
	\begin{frame}{Table of contents}
		\setbeamertemplate{section in toc}[sections numbered]
		\tableofcontents[hideallsubsections]
		\note[item]{\todo[inline]{COMPROBAR QUE FUNCIONA BIEN CON UNA SEGUNDA PANTALLA, QUE SE MUESTRAN LAS NOTAS EN UNA Y LA PRESENTACIÓN EN OTRA.}}
		\note[item]{\todo[inline]{COMPROBARLO CON LAS PANTALLAS DE LA PRESENTACIÓN (MI ORDE NO TIENE VGA, PREGUNTAR SI HAY HDMI Y LLEVAR EL MIO POR SI ACASO)}}
		\note[item]{\todo[inline]{EN LA TERMINAL, \$: pdfpc main.pdf - -notes=right}}	
		\note[item]{\todo[inline]{https://bugs.kde.org/show\_bug.cgi?id=152585}}
		\note[item]{\todo[inline,color=pink]{ESTOY COMPILANDO CON LUALATEX, PARA PODER USAR LA FUENTE: fira. con xelatex no me funciona (se supone que debería) y con pdflatex es una letra diferente(NO CONSIGUE CARGAR BIEN fira). si me gusta la letra, seguir compilando con lualatex, sino con pdflatex que es la letra de siempre (aunque me gusta mas FIRA).}}
		
	\end{frame}

\section{Introducción}


\begin{frame}{Motivación y objetivos I}
\vspace{-2em}
Las redes definidas por software (SDN) surgen a principios de 2010 \alert{por necesidad}:
\begin{itemize}
	\item La mayoría de las redes tradicionales fueron diseñadas para aplicaciones cliente-servidor que se ejecutan en una infraestructura no virtualizada.
\end{itemize}

SDN se ha establecido en la adultez temprana como un producto conocido.

Es una realidad que muchas de las empresas y proveedores de servicios de todo el mundo ya han adoptado.


\note{\large \vfill
	\begin{center}
	\begin{enumerate}
		\item NO son la solución a un problema sin resolver sino que resuelven de una forma mas eficiente que las soluciones tradicionales. 		
		\vspace{2em}
		\item SDN ha crecido más allá de su adolescencia y euforia prematura...	
		\vspace{2em}	
		\item NO es una próxima novedad en el horizonte de la creación de redes...	
		\vspace{2em}
		\vfill
	\end{enumerate}
	\end{center}}
\end{frame}


\begin{frame}{Motivación y objetivos II}
\vspace{-2em}
\textit{Rohit Mehra} y \textit{Brad Casemore} en su previsión sobre SDN publicada en 2016:
\vspace{1.3em}

\textit{\say{Virtualization, cloud, mobility, and now the Internet of Things (IoT) have exposed the limitations of traditional network architectures and operational models}}


\note[item]{\todo[inline]{VER SI AÑADO ESTA DIAPO O NO, DECIDIRLO CUANDO TENGA TODAS LAS DIAPOS HECHAS, PARA VER SI SON DEMASIADAS. OCUPARÍA 15 SEGUNDOS EXPLICAR ESTA DIAPO, POR TIEMPO NO HABRÍA PROBLEMA}}
\end{frame}


% DIAPOSITIVA DEL HISTOGRAMA SOBRE LA ADOPCIÓN DE SDN
\begin{frame}[fragile]{Moivación y objetivos III}
\begin{figure}
\centering
\pgfplotsset{
	select row/.style={
		x filter/.code={\ifnum\coordindex=#1\else\def\pgfmathresult{}\fi}
	}
}

\pgfplotstableread[header=false]{
	74 {Suma de las que emplean o emplearán}
	26 {No}
	25 {En un futuro, sin fecha}
	21 {En los próximos 2 años}
	28 {Sí}
}\datatable

\hspace*{-6em}
\begin{tikzpicture}[yscale=0.7,xscale=0.7] %TAMAÑO DE LA GRÁFICA
\begin{axis}[
title= Adopción de las redes SDN en las empresas IT en 2016.,
xbar, bar shift=0pt,
enlarge y limits=0.2,
xmin=0,
ytick={0,...,5},
yticklabel style={text width=3cm,align=right},
yticklabels from table={\datatable}{1},
yticklabel style={align=right},
% xmajorgrids = true, %CON ESTO DESCOMENTADO MUESTRA UN GRID, SINO SOLO LAS BARRAS, SIN GRID.
bar width=8mm, 
width=12cm, height=8.5cm,  %CAMBIAR TAMAÑO DE LA GRAFICA, HORIZONTAL Y VERTICAL, HEIGHT PARA CAMBIAR EL  ESPACIO ENTRE BARRAS
xlabel={\% de las empresas IT encuestadas},
nodes near coords={\pgfmathprintnumber\pgfplotspointmeta\%},
nodes near coords align={horizontal}, 
]

\pgfplotsinvokeforeach{0,...,5}{
	\addplot table [select row=#1, y expr=#1] {\datatable};
}
\end{axis}
\end{tikzpicture}

\vspace{1.5em}\source{\href{https://www.channelinsider.com/networking/slideshows/enterprise-interest-in-sdn-adoption-picks-up-steam.html}{Channel Insider Networking - Michael Vizard}}

\note{\large \vfill
	\begin{center}
		\begin{enumerate}
			\item Debido a la creciente demanda en las redes, en estos años se ha visto una evolución en el mercado de SDN.	
			\vspace{2em}	
			\item Es por esto y por el TFG DE RUBEN que surge el proyecto.	
			\vspace{2em}
			\vfill
		\end{enumerate}
\end{center}}

\end{figure}
\end{frame}





\begin{frame}{Motivación y objetivos IV}
\begin{alertblock}{OBJETIVOS}
\begin{itemize}
	\item[\textbf{>>}] Exponer dos casos de uso real de las redes SDN.
	\item[\textbf{>>>>}] Aplicar técnicas de ingeniería de tráfico en esos casos.
	\item[\textbf{>>>>>>}] Conocer el controlador Ryu y 
\end{itemize}
\end{alertblock}

\note[item]{\todo[inline]{COMPROBAR QUE LOS TITULOS DE ESTAS DIAPOS ESTÁN BIEN: I, II, III, IV, V, ...}}
\end{frame}





\section{Ámbito de trabajo}
\section{Definición del escenario de aplicación}
\section{Routing multicamino con balanceador de carga}
\section{Implementación}
\section{Conclusiones y líneas futuras}







\begin{frame}[fragile]{Metropolis}

The \themename theme is a Beamer theme with minimal visual noise
inspired by the \href{https://github.com/hsrmbeamertheme/hsrmbeamertheme}{\textsc{hsrm} Beamer
Theme} by Benjamin Weiss.

Enable the theme by loading

\begin{verbatim}    \documentclass{beamer}
\usetheme{metropolis}\end{verbatim}

Note, that you have to have Mozilla's \emph{Fira Sans} font and XeTeX
installed to enjoy this wonderful typography.
\end{frame}
\begin{frame}[fragile]{Sections}
Sections group slides of the same topic

\begin{verbatim}    \section{Elements}\end{verbatim}

for which \themename provides a nice progress indicator \ldots
\end{frame}


\begin{frame}{Metropolis title formats}
\themename supports 4 different title formats:
\begin{itemize}
\item Regular
\item \textsc{Small caps}
\item \textsc{all small caps}
\item ALL CAPS
\end{itemize}
They can either be set at once for every title type or individually.
\end{frame}

{
\metroset{titleformat frame=smallcaps}
\begin{frame}{Small caps}
This frame uses the \texttt{smallcaps} title format.

\begin{alertblock}{Potential Problems}
Be aware that not every font supports small caps. If for example you typeset your presentation with pdfTeX and the Computer Modern Sans Serif font, every text in small caps will be typeset with the Computer Modern Serif font instead.
\end{alertblock}
\end{frame}
}

{
\metroset{titleformat frame=allsmallcaps}
\begin{frame}{All small caps}
This frame uses the \texttt{allsmallcaps} title format.

\begin{alertblock}{Potential problems}
As this title format also uses small caps you face the same problems as with the \texttt{smallcaps} title format. Additionally this format can cause some other problems. Please refer to the documentation if you consider using it.

As a rule of thumb: just use it for plaintext-only titles.
\end{alertblock}
\end{frame}
}

{
\metroset{titleformat frame=allcaps}
\begin{frame}{All caps}
This frame uses the \texttt{allcaps} title format.

\begin{alertblock}{Potential Problems}
This title format is not as problematic as the \texttt{allsmallcaps} format, but basically suffers from the same deficiencies. So please have a look at the documentation if you want to use it.
\end{alertblock}
\end{frame}
}


\begin{frame}[fragile]{Typography}
\begin{verbatim}The theme provides sensible defaults to
\emph{emphasize} text, \alert{accent} parts
or show \textbf{bold} results.\end{verbatim}

\begin{center}becomes\end{center}

The theme provides sensible defaults to \emph{emphasize} text,
\alert{accent} parts or show \textbf{bold} results.
\end{frame}

\begin{frame}{Font feature test}
\begin{itemize}
\item Regular
\item \textit{Italic}
\item \textsc{Small Caps}
\item \textbf{Bold}
\item \textbf{\textit{Bold Italic}}
\item \textbf{\textsc{Bold Small Caps}}
\item \texttt{Monospace}
\item \texttt{\textit{Monospace Italic}}
\item \texttt{\textbf{Monospace Bold}}
\item \texttt{\textbf{\textit{Monospace Bold Italic}}}
\end{itemize}
\end{frame}

\begin{frame}{Lists}
\begin{columns}[T,onlytextwidth]
\column{0.33\textwidth}
Items
\begin{itemize}
\item Milk \item Eggs \item Potatoes
\end{itemize}

\column{0.33\textwidth}
Enumerations
\begin{enumerate}
\item First, \item Second and \item Last.
\end{enumerate}

\column{0.33\textwidth}
Descriptions
\begin{description}
\item[PowerPoint] Meeh. \item[Beamer] Yeeeha.
\end{description}
\end{columns}
\end{frame}
\begin{frame}{Animation}
\begin{itemize}[<+- | alert@+>]
\item \alert<4>{This is\only<4>{ really} important}
\item Now this
\item And now this
\end{itemize}
\end{frame}
\begin{frame}{Figures}
\begin{figure}
\newcounter{density}
\setcounter{density}{20}
\begin{tikzpicture}
\def\couleur{alerted text.fg}
\path[coordinate] (0,0)  coordinate(A)
++( 90:5cm) coordinate(B)
++(0:5cm) coordinate(C)
++(-90:5cm) coordinate(D);
\draw[fill=\couleur!\thedensity] (A) -- (B) -- (C) --(D) -- cycle;
\foreach \x in {1,...,40}{%
\pgfmathsetcounter{density}{\thedensity+20}
\setcounter{density}{\thedensity}
\path[coordinate] coordinate(X) at (A){};
\path[coordinate] (A) -- (B) coordinate[pos=.10](A)
-- (C) coordinate[pos=.10](B)
-- (D) coordinate[pos=.10](C)
-- (X) coordinate[pos=.10](D);
\draw[fill=\couleur!\thedensity] (A)--(B)--(C)-- (D) -- cycle;
}
\end{tikzpicture}
\caption{Rotated square from
\href{http://www.texample.net/tikz/examples/rotated-polygons/}{texample.net}.}
\end{figure}
\end{frame}
\begin{frame}{Tables}
\begin{table}
\caption{Largest cities in the world (source: Wikipedia)}
\begin{tabular}{@{} lr @{}}
\toprule
City & Population\\
\midrule
Mexico City & 20,116,842\\
Shanghai & 19,210,000\\
Peking & 15,796,450\\
Istanbul & 14,160,467\\
\bottomrule
\end{tabular}
\end{table}
\end{frame}
\begin{frame}{Blocks}
Three different block environments are pre-defined and may be styled with an
optional background color.

\begin{columns}[T,onlytextwidth]
\column{0.5\textwidth}
\begin{block}{Default}
Block content.
\end{block}

\begin{alertblock}{Alert}
Block content.
\end{alertblock}

\begin{exampleblock}{Example}
Block content.
\end{exampleblock}

\column{0.5\textwidth}

\metroset{block=fill}

\begin{block}{Default}
Block content.
\end{block}

\begin{alertblock}{Alert}
Block content.
\end{alertblock}

\begin{exampleblock}{Example}
Block content.
\end{exampleblock}

\end{columns}
\end{frame}
\begin{frame}{Math}
\begin{equation*}
e = \lim_{n\to \infty} \left(1 + \frac{1}{n}\right)^n
\end{equation*}
\end{frame}
\begin{frame}{Line plots}
\begin{figure}
\begin{tikzpicture}
\begin{axis}[
mlineplot,
width=0.9\textwidth,
height=6cm,
]

\addplot {sin(deg(x))};
\addplot+[samples=100] {sin(deg(2*x))};

\end{axis}
\end{tikzpicture}
\end{figure}
\end{frame}
\begin{frame}{Bar charts}
\begin{figure}
\begin{tikzpicture}
\begin{axis}[
mbarplot,
xlabel={Foo},
ylabel={Bar},
width=0.9\textwidth,
height=6cm,
]

\addplot plot coordinates {(1, 20) (2, 25) (3, 22.4) (4, 12.4)};
\addplot plot coordinates {(1, 18) (2, 24) (3, 23.5) (4, 13.2)};
\addplot plot coordinates {(1, 10) (2, 19) (3, 25) (4, 15.2)};

\legend{lorem, ipsum, dolor}

\end{axis}
\end{tikzpicture}
\end{figure}
\end{frame}
\begin{frame}{Quotes}
\begin{quote}
Veni, Vidi, Vici
\end{quote}
\end{frame}

{%
\setbeamertemplate{frame footer}{My custom footer }

\begin{frame}[fragile]{Frame footer}
\themename defines a custom beamer template to add a text to the footer. It can be set via
\begin{verbatim}\setbeamertemplate{frame footer}{My custom footer}\end{verbatim}
\end{frame}
}

\begin{frame}{References}
Some references to showcase [allowframebreaks] \cite{knuth92,ConcreteMath,Simpson,Er01,greenwade93}
\end{frame}

\begin{frame}{Summary}

Get the source of this theme and the demo presentation from

\begin{center}\url{github.com/matze/mtheme}\end{center}

The theme \emph{itself} is licensed under a
\href{http://creativecommons.org/licenses/by-sa/4.0/}{Creative Commons
Attribution-ShareAlike 4.0 International License}.

\begin{center}\ccbysa\end{center}

\end{frame}

\begin{frame}[standout]
Questions?
\end{frame}

\appendix

\begin{frame}[fragile]{Backup slides}
Sometimes, it is useful to add slides at the end of your presentation to
refer to during audience questions.

The best way to do this is to include the \verb|appendixnumberbeamer|
package in your preamble and call \verb|\appendix| before your backup slides.

\themename will automatically turn off slide numbering and progress bars for
slides in the appendix.
\end{frame}

\begin{frame}[allowframebreaks]{References}

\bibliography{demo}
\bibliographystyle{abbrv}

\end{frame}

\end{document}
